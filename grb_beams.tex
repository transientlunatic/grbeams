\documentclass[twocolumn,nofootinbib]{revtex4}
%\usepackage{amsmath}
%\usepackage{iopams}
\usepackage{amsmath}
\usepackage{hyperref}
\usepackage{amssymb}
\usepackage{color}
\usepackage{epsfig}
\usepackage{latexsym}
%\usepackage{wasysym}
\usepackage{comment}
%\usepackage{graphicx}
%\usepackage{psfrag}


\newcommand{\gw}{gravitational wave }
\newcommand{\gws}{gravitational waves }
\newcommand{\subgw}{_{\textrm{\scriptsize{GW}}}}
\newcommand{\ee}[1]{\!\times\!10^{#1}}
\newcommand{\prob}{{\rm Pr}}
\newcommand{\grbrate}{{{\mathcal R}_{\mathrm{grb}}}}
\newcommand{\cbcrate}{{{\mathcal R}_{\mathrm{cbc}}}}
\newcommand{\diff}{{\mathrm d}}
\newcommand{\rhostar}{{\rho^*}}

\begin{document}

\title{Constraints On Short, Hard Gamma-Ray Burst Beaming Angles From
Gravitational Wave Observations}
\author{James Clark and Ik Siong Heng}
\date{\today}

\begin{abstract}
Apologies in advance for inconsistent conditioning statements in probabilities
\dots
\end{abstract}

\maketitle

\section{Introduction}

It is common in the literature to draw inferences on the rate of binary
coalescence $\cbcrate$, given some estimate for the beaming angle $\theta$ and
the observed rate of sGRBs $\grbrate$.  In this work, we investigate what
statements can \emph{currently} be made on the beaming angle itself using the
upper limits placed on $\cbcrate$ from all-sky, all-time \gw searches and
explore the potential for direct inference of sGRB  beaming angles in the
advanced detector era.

\section{Limits On sGRB Beaming Angles From Past Gravitational Wave Searches}

\subsection{Complete Sky-Coverage \& Known Observed GRB Rate}
Assuming all-sky Gamma-ray coverage and that all compact binary coalescence
events result in a short-hard gamma-ray burst, the rate of binary coalescences
is,
%
\begin{equation}\label{eq:rate2angle}
\cbcrate=\frac{\grbrate}{1-\cos \theta},
\end{equation}
where $\theta$ is the beaming angle of the outflow from the GRB and $\grbrate$
is the \emph{observed} sGRB rate.  We take
$\grbrate=10$\,Gpc$^{-3}$\,yr$^{-1}$~\cite{nakar-2007,Dietz11}.
%
Inferences of the GRB beaming angle may then be drawn from the posterior
probability density on the beaming angle, related to that on the rate posterior
via,
%
\begin{eqnarray}
p(\theta|D,I) & = & p(\cbcrate|D,I) \left|\frac{\diff \cbcrate}{\diff
\theta}\right| \\
\label{eq:rate2angleProb}
& = & p(\cbcrate|D,I) \times \left| \frac{\grbrate\sin\theta}{(1-\cos \theta)^2} \right|,
\end{eqnarray}
%
where the Jacobian is computed from equation~\ref{eq:rate2angle}, $D$ represents
our \gw observations and we explicitly include the conditioning information $I$
to remind us of the assumptions in the analysis.

Following~\cite{Biswas09,BradyFairhurst08}, the posterior on the binary
coalescence rate may be determined from the loudest event in the \gw
analysis.  Specifically, for a foreground event rate due to  binary coalescence
$\cbcrate$, the probability of obtaning no events with ranking statistic $\rho$
greater than the observed loudested event $\rhostar$ is,
%
\begin{equation}
P_F(\rhostar | \cbcrate, C_L, T) = e^{-\cbcrate C_L(\rhostar) T},
\end{equation}
%
where $C_L(\rhostar)$ is the total luminosity to which the search is sensitive
and $T$ is the duration of the search.  The overall probability of obtaining
zero events with ranking statistic $\rho>\rhostar$ is the product of obtaining
no events from foreground \emph{and} the probability of obtaining no events from
the background in the detector, denoted $P_B(\rhostar)$,
%
\begin{equation}
P(\rhostar|\cbcrate,I) = P_B(\rhostar|I)e^{-\cbcrate C_L(\rhostar) T}
\end{equation}
%
Using a uniform prior on $\cbcrate$ and inverting the overall probability with
Bayes' theorem, we arrive at,
%
\begin{equation}\label{eq:loudestEventPosterior}
p(\cbcrate | \hat{\epsilon}, \hat{\Lambda}) =
\frac{\hat{\epsilon}}{1+\hat{\Lambda}}
\left(1 + \cbcrate \hat{\epsilon}\hat{\Lambda}\right) e^{-\cbcrate \hat{\epsilon}},
\end{equation}
%
where, for notational and computational convenience and for consistency
with~\cite{Biswas09}, we set,
%
\begin{equation}
\epsilon = C_L(\rhostar) T
\end{equation}
%
The quantity $\hat{\Lambda}$ measures the relative probability of detecting an
event with ranking statistic $x$ due to \gws versus the probability of an
equally loud event arising in the background distribution.  For now, let us
focus attention on the interpretation of existing upper limits from past \gw
searches where no \gw signal has been observed and the loudest event is
umabiguously due to background noise fluctuations.  That is, the limit in which
$\hat{\Lambda} \rightarrow 0$.  In this case, the posterior on the rate goes to,
%
\begin{equation}
\label{eq:nullPosterior}
p(\cbcrate | \hat{\epsilon}, \hat{\Lambda} \rightarrow 0) = \hat{\epsilon} e^{-\cbcrate
\hat{\epsilon}} 
\end{equation}
%
Now, our objective is to compute a posterior on the GRB beaming angle via
equation~\ref{eq:rate2angleProb}, for which we require the (unpublished) numerical
value of $\hat{\epsilon}$.  To proceed, we note that the reported 90\%
confidence rate upper limits in~\cite{S6lowmass} are found by solving for
$\cbcrate^{90\%}$ in,
%
\begin{eqnarray}
0.9 & = &\int_0^{\cbcrate^{90\%}} p(\cbcrate | \hat{\epsilon}, \hat{\Lambda} \to
0)~\diff \cbcrate \nonumber \\
& = & 1 -
\left[1+\frac{\cbcrate^{90\%}\hat{\epsilon}\hat{\Lambda}}{1+\hat{\Lambda}}\right]
e^{-\cbcrate^{90\%} \hat{\epsilon}}.
\label{eq:rateIntegral}
\end{eqnarray}
%
So, \emph{given} $\cbcrate^{90\%}$, we simply solve
equation~\ref{eq:rateIntegral} for $\hat{\epsilon}$ to reconstruct the rate
posterior obtained in the search~\textcolor{red}{NOTE: I only just noticed
that~\cite{Biswas09} actually has the expression for this.  Existing results do
the calculation for $\hat{epsilon}$ iteratively but this is an unnecessary (but
fun) brute force approach.  I still need to make that change.}

The reconstructed rate posterior, $p(\cbcrate | \hat{\epsilon}, \hat{\Lambda}
\to 0)$ and the {\bf upper limit for BNS} is shown in
figure~\ref{fig:reconstructedRatePosterior}.  Note that this procedure
necessarily confines our jet angle inferences based on progenitor systems for
which the rate upper limits are available; we are not free to choose our own
mass configurations.

Finally, figure~\ref{fig:jetPosterior} shows the resulting posterior on the GRB
beaming angle, using the transformation in equation~\ref{eq:rate2angleProb} and
our results so far.  \textcolor{red}{Informal}: To interpret this posterior and
hence obtain a limit on the range of possible beaming angles, we need to
consider the question we're asking.  I (James) think this should probably be,
`what is the smallest jet angle that produces a result consistent with our
observations?'.  That is, we want a \emph{lower} limit.  Why not the
\emph{largest} jet angle?  I think that would involve allowing infinitely small
angles which doesn't really make sense.  I can't quite put my finger on it,
though.

Now, the quantity of interest is the inferred \emph{lower}
limit on the jet angle~$\theta^{90\%}$:
%
\begin{equation}
0.9 = \int_{\theta^{90\%}}^{\infty}p(\theta | \cbcrate^{90\%})~\diff \theta
\end{equation}
%
This is indicated with the vertical red line in figure~\ref{fig:jetPosterior}.
We find $\theta^{90\%}=3.3^{\circ}$.


\begin{figure}
\includegraphics{rate_posterior_s6UL.eps}
\caption{Rate posterior for S6/VSR2,3 Low-mass Search For Compact Binary
Coalescence.\label{fig:reconstructedRatePosterior}}
\end{figure}

\begin{figure}
\includegraphics{jet_angle_posterior_s6UL.eps}
\caption{Jet angle posterior derived from S6/VSR2,3 Low-mass Search For Compact Binary
Coalescence posterior / upper limit on compact binary coalescence
rate.\label{fig:jetPosterior}}
\end{figure}

\subsection{Incomplete Sky-Coverage \& Unknown GRB Rate}
Marginalise over some stuff for the case of an `impure' GRB sample (unknown
fraction of CBCs).
\\

\subsection{Astrophysical Interpretation \& Comparison With Other Limits}
The comparison with other limits is straightforward.  I think it looks pretty
consistent with Dietz, Holtz etc (but should double check).

More importantly, however,  we've demonstrated how to get a limit on the beaming
angle.  That's all well and good but we need some astrophysical interpretation,
really.  The most obvious thing (to James) is that the beaming angle is really a
proxy for the Lorentz factor of the outflow (I think):
%
\begin{equation}
\theta \sim \frac{1}{\Gamma}.
\end{equation}
%
So, what \emph{physics} of the progenitor can we constrain with this?

\section{Inferences On Beaming Angles Expected From Future Detections}
I'm less sure how to address this but I could imagine having a) a mock result
based on the Big Dog or b) a collection of mock results for e.g., a measured
non-zero rate after x years of observation (say).  Anything here is going to be
pretty speculative since we don't know about satellites.  In fact, maybe this is
an opportunity to devise a simple figure of merit to state what sort of
sky-coverage is required in each of the timelines in the observing scenarios
document to measure the jet angle to some accuracy?

\section{Conclusion}

\bibliography{grb_beams}

\end{document}
