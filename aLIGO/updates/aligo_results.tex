\documentclass{beamer}
\setbeamertemplate{navigation symbols}{}

\usepackage{beamerthemeshadow}
\setbeamertemplate{caption}[numbered]

\hypersetup{colorlinks}

\def\gw#1{gravitational wave#1 (GW#1)\gdef\gw{GW}}
\def\ns#1{neutron star#1 (NS#1)\gdef\ns{NS}}

\newcommand{\grbrate}{{{\mathcal R}_{\mathrm{grb}}}}
\newcommand{\cbcrate}{{{\mathcal R}}}
\newcommand{\red}[1]{{\color{red}{#1}}}

\begin{document}
\title{GRB-beams \& aLIGO Scenarios}
%\subtitle{Burst Call Oct 8$^{\text{th}}$ 2014}  
\author{James A. Clark}
%\institute{Georgia Institute Of Technology}
\date{} 

\begin{frame}[plain]
\titlepage
\end{frame}

%\begin{frame}\frametitle{Table of contents}\tableofcontents
%\end{frame} 

\section{Recap}
\begin{frame}
\frametitle{Recap}
Key equation:
\begin{equation}\label{eq:rate2angle}
\grbrate=\epsilon\cbcrate(1-\cos \theta),
\end{equation}
where $\theta$ is the \emph{mean} of the distribution of GRB beaming angles.

Explicitly: the distribution of GRB beaming angles can be broad, but if we only
look at the relative GRB and BNS rates ($\grbrate$, $\cbcrate$, respectively),
we only probe the mean value of that distribution

\end{frame}

\begin{frame}
    \frametitle{Beaming angle \& rates}
    {\tt monte-carlo demonstration of individual thetas on rates}
\end{frame}

\section{GRB `Injections'}

\begin{frame}

    \frametitle{GRB `Injections'}
    \begin{itemize}
        \item 
    \end{itemize}

\end{frame}


\end{document}
